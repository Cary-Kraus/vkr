\abstract{РЕФЕРАТ}

Объем работы равен \formbytotal{lastpage}{страниц}{е}{ам}{ам}. Работа содержит \formbytotal{figurecnt}{иллюстраци}{ю}{и}{й}, \formbytotal{tablecnt}{таблиц}{у}{ы}{}, \arabic{bibcount} библиографических источников и \formbytotal{числоПлакатов}{лист}{}{а}{ов} графического материала. Количество приложений – 2. Графический материал представлен в приложении А. Фрагменты исходного кода представлены в приложении Б.

Перечень ключевых слов: игровой движок, Point-and-Click, приключенческие игры, Windows Forms, головоломки, симуляторы, ООП, объект, класс, абстракция, наследование, полиморфизм, форма, десктопное приложение, инкапсуляция, абстрактный класс, список, игрок, компонент, модуль, сущность, анимация, метод, предмет, локация, пользователь, pixel art.

Объектом разработки является движок игры, с помощью которого реализована приключенческая игра в формате point-and-click.

Целью выпускной квалификационной работы является создание программной системы, которая обеспечивает удобное и эффективное создание приключенческих игр.

В рамках работы будет разработан игровой движок, позволяющий дизайнерам и разработчикам создавать уникальные и захватывающие игровые миры. Кроме того, будет предусмотрена возможность интеграции различных элементов игры, таких как графика, звук, анимация, сценарии и управление персонажами. Основной задачей работы является улучшение процесса создания приключенческих игр и повышение качества готового продукта.

При разработке сайта использовалась система управления версий "Git.hub">.

\selectlanguage{english}
\abstract{ABSTRACT}
  
The volume of work is \formbytotal{lastpage}{page}{}{s}{s}. The work contains \formbytotal{figurecnt}{illustration}{}{s}{s}, \formbytotal{tablecnt}{table}{}{s}{s}, \arabic{bibcount} bibliographic sources and \formbytotal{числоПлакатов}{sheet}{}{s}{s} of graphic material. The number of applications is 2. The graphic material is presented in annex A. The layout of the site, including the connection of components, is presented in annex B.

List of keywords: commercial website, System, CMS, Bitrix, Joomla, additive technologies, 3D printers, services, services, informatization, automation, information technology, web form, Apache, classes, database, component, module, entity, information block, method, content editor, administrator, user, web site.

The object of the research is the analysis of information technologies for the development of a production company's website.

The object of the development is the website of a company engaged in the production of 3D printers, the production of equipment for the creation of powders, software development and the organization of additive manufacturing centers.

The purpose of the final qualifying work is to create a software system that provides convenient and efficient development of adventure games.

In the process of creating the site, a game engine will be developed, allowing designers and developers to create unique and engaging game worlds. Additionally, the integration of various game elements such as graphics, sound, animation, scripts, and character control will be provided. The main goal of the project is to enhance the process of creating adventure games and improve the quality of the final product.

When developing the site, the content management system <<Git.hub>> was used.

\selectlanguage{russian}
