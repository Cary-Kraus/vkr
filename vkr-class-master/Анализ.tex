\section{Анализ предметной области}
\subsection{История и описание Point-and-click игр}

Point-and-click (point’n’click, point-n-click, с англ. — «укажи и щёлкни»)  это жанр видеоигр, где ключевым элементом игрового процесса является наведение курсора мыши на активные области и нажатие по ним. Чаще такие игры представлены в 2D, а перед игроком открывается область-локация, как правило, с видом сбоку. Пользователи могут отдавать приказы персонажу, собирать предметы, перемещать их и взаимодействовать с ними различным образом. Игры Point \& Click любят разбавлять всяческими головоломками, интересным сюжетом, диалогами, отличным саундтреком и необычным графическим оформлением.

Главным образом в качестве манипулятора для управления данным действием используется компьютерная мышь, однако могут быть задействованы аналоги либо заменители мыши (джойстик, клавиатура). Типичным примером point-and-click является использование мыши в гипертекстовом документе, где нажатие по ссылке инициирует переход в другую область документа или в другой документ.
Интерфейс point-and-click стал широко распространяться в компьютерных играх начиная с 1980-х годов с появлением мультимедийных домашних компьютеров (Amiga, Atari ST, IBM PC, Macintosh), которые уже могли поддерживать оконный интерфейс в своих операционных системах.

Одними из первых point-and-click стали использовать графические приключенческие игры как наиболее удобный способ взаимодействия пользователя с игровым миром. Этот метод произвёл небольшую революцию в данном жанре — произошёл основополагающий переход игры от первого лица к третьему, появился главный герой как обособленный игровой персонаж. Также наметился переход от командного режима управления персонажа (ввод глаголов-команд в текстовой строке) к управлению манипулятором. 
Самыми знаменитыми представителями жанра Point and Click можно назвать серии Syberia (Сибирь), The Curse of Monkey Island, Papers, Please, Botanicula и другие. Ниже вы можете найти самые лучшие игры с особенностью Point \& Click на ПК (PC), PlayStation, XBox, айфон и андроид.

\subsection{История и описание Приключенческих игр}

Приключенческая игра (англ. adventure game) или квест (от англ. quest) — один из основных жанров компьютерных игр, представляющий собой интерактивную историю с главным героем, управляемым игроком. Важнейшими элементами игры в жанре квеста являются собственно повествование и исследование мира, а ключевую роль в игровом процессе играет решение головоломок и задач, требующих от игрока умственных усилий. Такие характерные для других жанров компьютерных игр элементы, как бои, экономическое планирование и задачи, требующие от игрока скорости реакции и быстрых ответных действий, в квестах сведены к минимуму или вовсе отсутствуют[1]. Игры, объединяющие в себе характерные признаки квестов и жанра action, выделяют в отдельный жанр — action-adventure.

Одним из видов приключенческих игр является графические квесты. Первые графические квесты появились ещё для 8-битных домашних компьютеров в начале 1980-х. Однако по-настоящему «графическими» они стали лишь в тот момент, когда произошёл отказ от текстового интерфейса и переход к так называемому «point-and-click» (то есть управлению с помощью указателя посредством стрелок клавиатуры, джойстика или мыши), появившемуся в 1985. Одними из популярных игр этого поджанра являются серии игр Monkey Island и Space Quest.

Развитие вычислительной техники и появление домашних компьютеров, отличающихся развитой графической системой (таких, как Apple II) послужило толчком к появлению индустрии компьютерных игр. Соответственно, жанр квестов также получил дальнейшее развитие. Квесты приобрели первые графические иллюстрации происходящего, которые поначалу были чисто декоративными. Зачастую графика была примитивной (например, векторная с заливкой) и это не требовало больших расходов памяти. Игрок по-прежнему управлял действиями персонажа, вводя последовательности команд на клавиатуре, а изображение было статичным и служило только для стимулирования воображения играющего.

Первые дискуссии о целесообразности применения графики в приключенческих играх относятся к 1983—1984 годам. Основным аргументом сторонников графики был: «Ничего в этом страшного нет, поскольку с одной стороны игры становятся привлекательнее, а расход памяти на графику компенсируется тем, что можно сэкономить на текстовом описании локаций». Противники этого подхода предупреждали, что усиление роли графики может привести к упрощению игр и к вырождению жанра, но технически графическое улучшение выглядело прогрессивно. Со временем сторонники графики победили и начали постепенно вытеснять текстовые игры.

Следующим этапом развития приключенческих игр стало изменение интерактивного взаимодействия игрока с виртуальным миром, когда вместо текстового ввода игрок мог кликнуть в некоторую область экрана мышью и получить результат — герой двигался в указанную точку, использовался указанный предмет и т. д. Так отпала необходимость в текстовом описании локаций — игрок мог «прощелкать» по всем объектам на экране и получить комментарий-описание что это такое.

К началу 1990-х годов адвентюры полностью изменились по сравнению с их оригинальными предшественниками начала 1980-х. Постепенное упрощение интерфейса привело к тому, что все многообразие глаголов (идти, взять, поговорить, положить и т. д.) свелось к одному слову «использовать»: «принять аспирин» → «использовать аспирин», «открыть дверь» → «использовать ключ на двери». Играть в такие игры стало проще, но была утрачена цель первых классических адвентюр — «установление контакта с программой и исследование её словаря». Следствием стало то, что упрощение стало влиять на логику игровых механик. Если у игрока оказывалось ограниченное число предметов и малое число команд, то он для прохождения перебирал все возможные варианты. Для того, чтобы сохранить простоту и сформировать новый игровой вызов, в игры стали добавлять элементы Action. Например, в игре Indiana Jones and the Fate of Atlantis в одном из эпизодов нужно похитить каменный диск прямо из-под носа у его владельца. Для этого нужно выключить свет, надеть на голову простыню, включить фонарик и изобразить из себя привидение, и пока оцепеневший от ужаса хозяин будет в шоке, нужно незаметно стащить диск. Как следствие, смещение в сторону экшен и дальнейшее развитие в этом направлении стало стимулировать появление игр жанра приключенческий боевик.

К первой половине 1990-х изменения коснулись в основном улучшения реалистичности, что делалось более широком применении видео и звуковых технологий. Игры стали больше напоминать интерактивное кино, и для них стали привлекать профессиональных актёров.

\subsection{Золотой век и упадок квестов}

Соперничество двух крупнейших производителей квестов — Sierra On-Line и LucasArts благотворно сказывалось на самом жанре. В период с 1990 по 1998 год были выпущены лучшие игры из основных серий обеих компаний.

В очередной раз технический прогресс благотворно сказался на жанре компьютерных игр — с появлением звуковых карт появилась первая музыка, соответствующая атмосфере игры, а также озвучивание действий и событий. А с появлением таких ёмких носителей информации, как компакт-диск, стало возможным озвучивание диалогов персонажей.

Благодаря всему этому этот отрезок времени принято считать золотой эпохой графических квестов.

Однако появление первых графических акселераторов, а вслед за ними и первых трёхмерных игр послужило закатом эпохи квестов. Рынок требовал игры, демонстрирующей все возможности новых компьютеров.

Попытки сделать трёхмерные квесты имели лишь ограниченный успех, от такого внедрения технологии было больше вреда, чем пользы. В большинстве трёхмерных квестов был вид от третьего лица, и низкополигональные персонажи изображались поверх неподвижного двумерного фона. Неудачно расположенные (нередко неподвижные) точки обзора («камеры») дезориентировали игрока. Кроме того, с отказом от удобного и ставшего привычным режима point-and-click (с англ. — «укажи и щёлкни») играющему приходилось управлять персонажем с помощью стрелок клавиатуры, наблюдая зачастую, как тот бежит на месте, натолкнувшись на препятствие или невидимую стену. Существовали, правда, и полностью трёхмерные квесты с видом от первого лица.

Многие поклонники таких классических квестов, как Day of the Tentacle и Space Quest, не смогли принять эти трёхмерные новшества, а любители спецэффектов, как и прежде, обходили «занудные квесты» вниманием, предпочитая им всё более реалистичный 3D action и набирающий популярность жанр RPG. Испытывавшая финансовые трудности Sierra On-Line продала своё подразделение по производству приключенческих игр, а LucasArts переключилась на более прибыльные игры, посвящённые сражениям во вселенной Звёздных войн.


