\addcontentsline{toc}{section}{СПИСОК ИСПОЛЬЗОВАННЫХ ИСТОЧНИКОВ}

\begin{thebibliography}{9}

    \bibitem{javascript} Вигерс, К. Разработка требований к программному обеспечению / К. Вигерс. – Санкт-Петербург~: Вигерс, 2020. – 736 с. – ISBN 978-5-9775-3348-5. – Текст~: непосредственный.
    \bibitem{php} Космин, В. Основы научных исследований : учебное пособие / В. Космин. – Москва, 2020. – 238 с. – ISBN 978-5-369-01753-1. – Текст~: непосредственный.
    \bibitem{css} Рихтер, Д. Программирование на платформе Microsoft.NET. Framework 4.5 на языке C\# / Д. Рихтер. – Санкт-Петербург : Питер, 2017. – 896 с. – ISBN 978-5-496-00433-6. – Текст~: непосредственный.
    \bibitem{mysql}	Фаулер, Рефакторинг. Улучшение существующего кода / М. Фаулер. – Москва~: НТ Пресс, 2013. – 432 с. – ISBN 978-5-477-01174-2. – Текст~: непосредственный.
	\bibitem{html5}	Сидоренко, И. Дизайнер интерфейсов / И.	Сидоренко – Москва~: Вильямс, 2012. – 368 с. – ISBN 978-5-699-57580-0. – Текст~: непосредственный.
	\bibitem{htmlcss}	Хейлсберг, А. Язык программирования C\# / А. Хейлсберг. – Питер~: Эксмо, 2014. – 773 с. – ISBN 978-5-699-64193-2. – Текст~: непосредственный.
	\bibitem{bigbook}	Макфарланд, Д. Большая книга CSS / Д. Макфарланд. – Санкт-Петербург : Питер, 2012. – 560 с. – ISBN 978-5-496-02080-0. – Текст~: непосредственный.
	\bibitem{uchiru}	Лоусон, Б. Изучаем HTML5. Библиотека специалиста / Б. Лоусон, Р. Шарп. – Санкт-Петербург : Питер, 2013 – 286 с. – ISBN 978-5-459-01156-2. – Текст~: непосредственный.
	\bibitem{chaynik}	Мартин, Р. Чистая архитектура. Искусство разработки программного обеспечения / Р. Мартин – Санкт-Петербург : Питер, 2018 – 352 с. – ISBN 978-5-4461-0772-8. – Текст~: непосредственный.    
	\bibitem{22}	Титтел, Э. HTML5 и CSS3 для чайников / Э. Титтел, К. Минник. – Москва~: Вильямс, 2016 – 400 с. – ISBN 978-1-118-65720-1. – Текст~: непосредственный.    
	\bibitem{1231}	Белик, А. Г. Проектирование и архитектура программных систем / А. Г.  Белик, В. Н. Цыганенко. – Омск~: ОмГТУ, 2016 – 96 с. – ISBN 978-5-8149-2258-8. – Текст~: непосредственный.	
	\bibitem{htmlcss}	Хейлсберг, А. Язык программирования C\# / А. Хейлсберг. – Питер~: Эксмо, 2014. – 773 с. – ISBN 978-5-699-64193-2. – Текст~: непосредственный.
\end{thebibliography}
