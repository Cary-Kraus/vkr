\section{Анализ предметной области}
\subsection{История и описание Point-and-click игр}

Point-and-click (point’n’click, point-n-click, с англ. — «укажи и щёлкни»)  это жанр видеоигр, где ключевым элементом игрового процесса является наведение курсора мыши на активные области и нажатие по ним. Чаще такие игры представлены в 2D, а перед игроком открывается область-локация, как правило, с видом сбоку. Пользователи могут отдавать приказы персонажу, собирать предметы, перемещать их и взаимодействовать с ними различным образом. Игры Point \& Click любят разбавлять всяческими головоломками, интересным сюжетом, диалогами, отличным саундтреком и необычным графическим оформлением.

Главным образом в качестве манипулятора для управления данным действием используется компьютерная мышь, однако могут быть задействованы аналоги либо заменители мыши (джойстик, клавиатура). Типичным примером point-and-click является использование мыши в гипертекстовом документе, где нажатие по ссылке инициирует переход в другую область документа или в другой документ.
Интерфейс point-and-click стал широко распространяться в компьютерных играх начиная с 1980-х годов с появлением мультимедийных домашних компьютеров (Amiga, Atari ST, IBM PC, Macintosh), которые уже могли поддерживать оконный интерфейс в своих операционных системах.

Одними из первых point-and-click стали использовать графические приключенческие игры как наиболее удобный способ взаимодействия пользователя с игровым миром. Этот метод произвёл небольшую революцию в данном жанре — произошёл основополагающий переход игры от первого лица к третьему, появился главный герой как обособленный игровой персонаж. Также наметился переход от командного режима управления персонажа (ввод глаголов-команд в текстовой строке) к управлению манипулятором. 
Самыми знаменитыми представителями жанра Point and Click можно назвать серии Syberia (Сибирь), The Curse of Monkey Island, Papers, Please, Botanicula и другие. Ниже вы можете найти самые лучшие игры с особенностью Point \& Click на ПК (PC), PlayStation, XBox, айфон и андроид.

\subsection{История и описание Приключенческих игр}

Квесты, также известные как приключенческие игры, представляют собой ключевой жанр в мире компьютерных развлечений. Они погружают игрока в интерактивный рассказ, где он направляет действия протагониста. Основной упор в этих играх делается на сюжет и исследование игрового пространства, а также на решение различных головоломок, которые ставят перед игроком интеллектуальные задачи. В отличие от других жанров, где акцентируется внимание на боевых действиях, стратегическом планировании или быстроте реакции, квесты минимизируют эти элементы или исключают их полностью. Игры, сочетающие элементы квестов и экшена, выделяются в отдельную категорию — приключенческие боевики.

Графические квесты — это поджанр приключенческих игр, который начал своё развитие с появлением 8-битных домашних компьютеров в начале 80-х годов. Они стали настоящими графическими играми после отказа от текстового интерфейса и перехода к системе управления "point-and-click", что произошло в 1985 году. Примерами популярных игр этого поджанра являются серии Monkey Island и Space Quest.

Прогресс в области вычислительной техники и появление домашних компьютеров с продвинутой графикой, таких как Apple II, дали толчок к развитию индустрии компьютерных игр и, соответственно, квестов. Эти игры начали включать графические иллюстрации событий, которые изначально выполняли декоративную функцию. Графика часто была простой, например, векторной с заливкой, что позволяло экономить память. Управление персонажем осуществлялось через ввод команд с клавиатуры, а статичные изображения стимулировали воображение игроков.

Дебаты о целесообразности использования графики в приключенческих играх начались в 1983-1984 годах. Сторонники графики утверждали, что она делает игры более привлекательными и что экономия на текстовом описании локаций компенсирует расходы памяти на графику. Противники опасались, что усиление роли графики может упростить игры и привести к деградации жанра. В конечном итоге, сторонники графики взяли верх, и графические игры начали вытеснять текстовые.

Следующий этап в развитии жанра связан с изменением интерактивности: вместо текстового ввода команд игроки получили возможность взаимодействовать с миром через клики мышью. Это позволило отказаться от текстового описания локаций, так как игроки могли исследовать объекты, кликая по ним на экране.

К началу 90-х годов интерфейс адвентюр стал значительно проще, сократив множество команд до одной — "использовать". Это упрощение повлияло на логику игровых механик, так как игроки начали использовать метод проб и ошибок для прохождения игры, имея ограниченное количество предметов и команд. Чтобы сохранить простоту и добавить новые вызовы, разработчики стали включать элементы экшена в игры.

К середине 90-х игры стали более реалистичными благодаря видео и звуковым технологиям, приближаясь к интерактивному кино и привлекая профессиональных актёров для участия в них.

\subsection{Золотой век и упадок квестов}

Конкуренция между двумя гигантами индустрии приключенческих игр, Sierra On-Line и LucasArts, оказала благотворное влияние на развитие жанра. В период с 1990 по 1998 годы обе компании выпустили некоторые из своих самых знаковых игр.

Технологический прогресс вновь оказал положительное воздействие на жанр, внося инновации в звуковое сопровождение: появление звуковых карт позволило создать музыкальное оформление, соответствующее атмосфере игры, и озвучить действия и события. Появление компакт-дисков увеличило возможности для озвучивания диалогов персонажей. Этот период часто называют "золотым веком" графических квестов.

Однако введение первых графических ускорителей и появление трёхмерных игр предвещали закат эры квестов. Рынок начал требовать игры, которые бы демонстрировали возможности новейших компьютеров.

Попытки создать трёхмерные квесты встретились с ограниченным успехом, и часто такие технологические новшества приносили больше вреда, чем пользы. Во многих трёхмерных квестах использовался вид от третьего лица, где низкополигональные персонажи были наложены на статичный двумерный фон. Неудачно расположенные камеры часто дезориентировали игроков, а отказ от привычного режима "point-and-click" заставлял игроков управлять персонажами с помощью клавиш-стрелок, что приводило к тому, что персонажи застревали на месте, сталкиваясь с препятствиями или невидимыми стенами. Также существовали полностью трёхмерные квесты с видом от первого лица.

Многие фанаты классических квестов, таких как "Day of the Tentacle" и "Space Quest", не смогли принять эти трёхмерные изменения. Тем временем, любители спецэффектов продолжали игнорировать "скучные квесты", отдавая предпочтение более реалистичным 3D экшенам и популярным RPG. Sierra On-Line, испытывая финансовые трудности, продала своё подразделение по разработке приключенческих игр, а LucasArts сосредоточилась на более прибыльных проектах, связанных с битвами во вселенной "Звёздных войн".


