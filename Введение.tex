\section*{ВВЕДЕНИЕ}
\addcontentsline{toc}{section}{ВВЕДЕНИЕ}

В последние годы индустрия компьютерных игр переживает настоящий бум. Среди множества жанров особое место занимают приключенческие игры, которые позволяют игрокам погрузиться в увлекательные истории и исследовать разнообразные миры. Жанр приключенческих игр, особенно point-and-click, требует от разработчиков не только художественного видения, но и мощных инструментов для реализации их идей. Именно поэтому мною и бы выбран этот жанр.

Целью данной работы является разработка программной системы — игрового движка, предназначенного для создания приключенческих игр. Такой движок должен обладать гибкостью и функциональностью, позволяющей разработчикам воплощать в жизнь сложные игровые механики, интерактивные сценарии и качественную графику, необходимые для создания захватывающих приключений.

В данном введении будет представлен обзор существующих решений в области игровых движков, а также обоснование необходимости создания новой системы, специализированной под жанр приключенческих игр. Особое внимание будет уделено аспектам удобства использования, оптимизации процесса разработки и возможностям интеграции с современными графическими и аудио технологиями.

Дальнейшие разделы дипломной работы будут посвящены детальному описанию архитектуры предлагаемого игрового движка, его ключевых компонентов и принципов работы. Также будет рассмотрена методология разработки игр на основе предложенной системы, включая примеры реализации типичных задач, стоящих перед создателями приключенческих игр.

Завершение введения составит формулировка задач, которые предстоит решить в ходе работы над дипломным проектом, и ожидаемые результаты исследования.

Главной задачей программной системы является предоставление возможностей воплощения фантазий пользователя в игровой мир и получение удовольствия от процесса создания и/или прохождения игры.

\emph{Цель настоящей работы} – разработка программной системы для привлечения дизайнеров, художников, разработчиков игр и просто начинающих программистов, а также любителей приключенческих игр. Для достижения поставленной цели необходимо решить \emph{следующие задачи:}
\begin{itemize}
\item изучить игровой рынок;
\item разработать концептуальную модель программной системы;
\item спроектировать программную систему для разработки приключенческих игр;
\item реализовать полноценную программную систему для разработки приключенческих игр.
\end{itemize}

\emph{Структура и объем работы.} Отчет состоит из введения, 4 разделов основной части, заключения, списка использованных источников, 2 приложений. Текст выпускной квалификационной работы равен \formbytotal{page}{страниц}{е}{ам}{ам}.

\emph{Во введении} сформулирована цель работы, поставлены задачи разработки, описана структура работы, приведено краткое содержание каждого из разделов.

\emph{В первом разделе} на стадии описания технической характеристики предметной области приводится сбор информации об игровой индустрии, для которой осуществляется разработка программной системы. 

\emph{Во втором разделе} на стадии технического задания приводятся требования к разрабатываемой программной системе.

\emph{В третьем разделе} на стадии технического проектирования представлены проектные решения для игрового движка.

\emph{В четвертом разделе} приводится список классов и их методов, использованных при разработке движка, производится тестирование разработанной программной системы.

В заключении излагаются основные результаты работы, полученные в ходе разработки.

В приложении А представлен графический материал.
В приложении Б представлены фрагменты исходного кода. 
